\newglossaryentry{partFrac}
{
	name = {Partial Fractions},
	description = {
		Taking a rational expression and decomposing it into simpler rational expressions that we can add or subtract to get the original rational expression is called \textbf{partial fraction decomposition}.  Many integrals involving rational expressions can be done if we first do partial fractions on the integrand.}
}

\newglossaryentry{conSys}
{
	name = {Control System},
	description = {
		A control system is an interconnection of components forming a system configuration that will provide a desired system response.
	}
}
\newglossaryentry{openLoop}
{
	name = {open-loop control system},
	description = {
		An open-loop control system utilizes an actuating device to control the process
		directly without using feedback.
	}
}
\newglossaryentry{closedLoop}{
	name={closed-loop control system},
	description={A closed-loop control system uses a measurement of the output and feedback of
		this signal to compare it with the desired output (reference or command).
	}
}

\newglossaryentry{SISO}
{
	name = {Single Input Single Output},
	description = {
		In control engineering, a single-input and single-output (SISO) system is a simple single variable control system with one input and one output. SISO systems are typically less complex than multiple-input multiple-output (MIMO) systems. Frequency domain techniques for analysis and controller design dominate SISO control system theory. }
}

\newglossaryentry{MIMO}
{
	name = {Multiple Input Multiple Output},
	description = {
		In control engineering, systems with more than one input and/or more than one output are known as Multi-Input Multi-Output systems, or they are frequently known by the abbreviation MIMO. MIMO systems that are lumped and linear can be described easily with state-space equations.}
}
\newglossaryentry{LTI}
{
	name = {linear time-invariant},
	description = {
		\textbf{Linear time-invariant systems} (LTI systems) are a class of systems used in signals and systems that are both linear and time-invariant. Linear systems are systems whose outputs for a linear combination of inputs are the same as a linear combination of individual responses to those inputs. Time-invariant systems are systems where the output does not depend on when an input was applied. 
	}
}

\newglossaryentry{DC Motors}
{
	name = {DC Motors},
	description = {
	A \textbf{DC motor} is any of a class of rotary electrical machines that converts direct current electrical energy into mechanical energy. The most common types rely on the forces produced by magnetic fields. Nearly all types of DC motors have some internal mechanism, either electromechanical or electronic, to periodically change the direction of current flow in part of the motor. Covered in ELEC 370  --- Electromagnetic energy conversion
	}
}

\newglossaryentry{Op Amps}
{
	name = {Op Amps},
	description = {
		An \textbf{operational amplifier }  (often op-amp or opamp) is a DC-coupled high-gain electronic voltage amplifier with a differential input and, usually, a single-ended output. In this configuration, an op-amp produces an output potential (relative to circuit ground) that is typically hundreds of thousands of times larger than the potential difference between its input terminals.  Covered in ELEC 300  --- Electric Circuit II
	}
}
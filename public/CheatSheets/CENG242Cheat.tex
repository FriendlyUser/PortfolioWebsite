%% Added this section, to ha ve a list of algorithms in lwarp
\chapter{CENG242Cheat}
\todo{Remember to add more algorithms later}{Future Work}

\begin{algorithm}
\caption{Euclid’s algorithm}\label{euclid}
\begin{algorithmic}[1]
\Procedure{Euclid}{$a,b$}\Comment{The g.c.d. of a and b}
\State $r\gets a\bmod b$
\While{$r\not=0$}\Comment{We have the answer if r is 0}
\State $a\gets b$
\State $b\gets r$
\State $r\gets a\bmod b$
\EndWhile\label{euclidendwhile}
\State \textbf{return} $b$\Comment{The gcd is b}
\EndProcedure
\end{algorithmic}
\end{algorithm}



 \begin{algorithm}
	\caption{Merge Sort}
	\begin{algorithmic}[1]
		\Function{Merge}{$A,p,q,r$}\Comment{Where A - array, p - left, q - middle, r - right}
		
		\State ${n_1} = q - p + 1$
		\State ${n_2} = r - q$
		\State Let $L[1 \ldots {n_1} + 1]$ and $R[1 \ldots {n_2} + 1]$ be new arrays
		\For{$i = 1$ to ${n_1}$}
		\State $L[i] = A[p + i - 1]$
		\EndFor
		\For{$j = 1$ to ${n_2}$}
		\State $R[i] = A[q + j]$
		\EndFor
		\State $L[{n_1} + 1] =  \infty $
		\State $R[{n_2} + 1] =  \infty $
		\State $i = 1$
		\State $j = 1$
		\For{$k = p$ to $r$}
		\If {$L[i] < R[j]$}
		\State $A[k] = L[i]$
		\State $i = i + 1$
		\ElsIf {$L[i] > R[j]$}
		\State $A[k] = R[j]$
		\State $j = j + 1$
		\Else
		\State $A[k] = - \infty$ \Comment{We mark the duplicates with the largest negative integer}
		\State $j = j + 1$
		\EndIf
		\EndFor
		\EndFunction
		
	\end{algorithmic}
\end{algorithm}

\section{Basic Graph Definitions}
\begin{itemize}
	\item A graph $G = (V,E)$ consists of a finite set of {\em vertices} $V$
	and a finite set of {\em edges} $E$.
	\begin{itemize}
		\item {\em Directed graphs}: $E$ is a set of ordered pairs of vertices
		$(u,v)$ where $u,v \in V$ \\
		% \centerline{\epsfig{file=directed_example.eps,height=3.5cm}}
		\item {\em Undirected graph}: $E$ is a set of unordered pairs of
		vertices $\{u,v\}$ where $u,v \in V$ \\ \\
		% \centerline{\epsfig{file=undirected_example.eps,height=3cm}}
	\end{itemize}
	\item Edge $(u,v)$ is {\em incident} to $u$ and $v$
	\item {\em Degree} of vertex in undirected graph is the number of edges
	incident to it.
	\item {\em In (out) degree} of a vertex in directed graph is the number of
	edges entering (leaving) it.
	\item A {\em path} from $u_1$ to $u_2$ is a sequence of vertices
	$<u_1$=$v_0,v_1,v_2, \cdots , v_k$=$u_2>$ such that $(v_i,v_{i+1}) \in E$
	(or $\{v_i,v_{i+1}\} \in E$)
	\begin{itemize}
		\item We say that $u_2$ is {\em reachable} from $u_1$
		\item The {\em length} of the path is $k$
		\item It is a {\em cycle } if $v_0 = v_k$
	\end{itemize}
	
	
	\item An undirected graph is {\em connected /} if every pair of vertices are
	connected by a path
	\begin{itemize}
		\item The {\em connected components } are the equivalence classes of
		the vertices under the ``reachability'' relation. (All connected
		pair of vertices are in the same connected component).
	\end{itemize}
	\item A directed graph is {\em strongly connected } if every pair of
	vertices are reachable from each other
	\begin{itemize}
		\item The {\em strongly connected components } are the equivalence
		classes of the vertices under the ``mutual reachability'' relation. 
	\end{itemize}
	
	\vspace*{\baselineskip}
	
	\item Graphs appear all over the place in all kinds of applications, e.g:
	\begin{itemize}
		\item Trees $(\vert E \vert = \vert V \vert - 1)$
		\item Connectivity/dependencies (house building plans, WWW-page
		connections = internet graph)
	\end{itemize}
	\item Often the edges $(u,v)$ in a graph have weights $w(u,v)$, e.g.
	\begin{itemize}
		\item Road networks (distances)
		\item Cable networks (capacity)
	\end{itemize}
\end{itemize}


\subsection{Representation}
\begin{itemize}
	\item {\em Adjacency-list} representation:
	\begin{itemize}
		\item Array of $\vert V \vert$ list of edges incident to each
		vertex. \\
		
		Examples: \\
		\item Note: For undirected graphs, every edge is stored twice.
		\item If graph is weighted, a weight is stored with each edge.
	\end{itemize}
	
	
	\item {\em Adjacency-matrix} representation:
	\begin{itemize}
		\item $\vert V \vert \times \vert V \vert$ matrix $A$ where 
		\[ a_{ij} = \left\{ \begin{array}{ll}
		1        & \mbox{ if $(i,j) \in E$} \\
		0        & \mbox{ otherwise } \\
		\end{array}
		\right. \]
		
		Examples: \\
		\item Note: For undirected graphs, the adjacency matrix is
		symmetric along the main diagonal ($A^T = A$).
		\item If graph is weighted, weights are stored instead of one's.
	\end{itemize}
	\item Comparison of matrix and list representation: \\
	\begin{itemize}
		\item[]
		\begin{tabular}{l|l}
			Adjacency list & Adjacency matrix \\ \hline
			$O(\vert V \vert + \vert E \vert)$ space & $O(\vert V \vert ^2)$
			space \\
			Good if graph {\em sparse} $(\vert E \vert << \vert V
			\vert ^2)$ & Good if graph {\em dense} $(\vert E \vert \approx
			\vert V \vert ^2)$ \\
			No quick access to $(u,v)$ & $O(1)$ access to $(u,v)$ \\
		\end{tabular}
	\end{itemize}
	\item We will use adjacency list representation unless stated otherwise
	($O(|V|+|E|)$ space).
\end{itemize}


\section{Graph traversal}
\begin{itemize}
	\item There are two standard (and simple) ways of traversing all
	vertices/edges in a graph in a systematic way
	\begin{itemize}
		\item Breadth-first
		\item Depth-first
	\end{itemize}
	\item We can use them in many fundamental algorithms, e.g finding cycles,
	connected components, $\dots$
\end{itemize}

\subsection{Breadth-first search (BFS)}
\begin{itemize}
	\item Main idea:
	\begin{itemize}
		\item Start at some source vertex $s$ and visit,
		\item All vertices at distance 1,
		\item Followed by all vertices at distance 2,
		\item Followed by all vertices at distance 3,
		
		$\vdots$
	\end{itemize}
	\item BFS corresponds to computing {\em shortest path} distance (number of
	edges) from $s$ to all other vertices.
	\item To control progress of our BFS algorithm, we think about {\em coloring}
	each vertex
	\begin{itemize}
		\item {\em White } before we start,
		\item {\em Gray } after we visit the vertex but before we have
		visited all its adjacent vertices,
		\item {\em Black } after we have visited the vertex and all its
		adjacent vertices (all adjacent vertices are gray).
	\end{itemize}
	\item We use a queue $Q$ to hold all gray vertices---vertices we have seen
	but  are still not done with.
	\item We remember from which vertex a given vertex $v$ is colored gray
	-- i.e. the node that discovered $v$ first; this is called
	parent[$v$].
	\item Algorithm: \\ \\
	\fbox{
		\parbox{10cm}{
			{\sc BFS($s$) }
			\begin{itemize}
				\item[] color[$s$] = gray
				\item[] $d[s] = 0$
				\item[] ENQUEUE($Q,s$)
				\item[] WHILE $Q$ not empty DO
				\begin{itemize}
					\item[] DEQUEUE($Q,u$)
					\item[] FOR $(u,v)\in E$ DO
					\begin{itemize}
						\item[] IF color[$v$] = white THEN
						\item[] ~~~~color[$v$] = gray
						\item[] ~~~~$d[v] = d[u] + 1$
						\item[] ~~~~parent[$v$] = u
						\item[] ~~~~ENQUEUE($Q,v$)
						\item[] FI
						\item[] color[$u$] = black
					\end{itemize}
				\end{itemize}
				\item[] OD
			\end{itemize}
	}}
	\vspace*{\baselineskip}
	
	\item Algorithm runs in $O(\vert V \vert + \vert E \vert)$ time
	
	
	\item Example (for directed graph): \\
	\item Note:
	\begin{itemize}
		\item parent[$v$] forms a tree; {\em BFS-tree}.
		\item $d[v]$ contains length of shortest path from $s$ to
		$v$. (Prove by induction)
		\item We can use parent[$v$] to find the shortest path from $s$ to a
		given vertex.
		%       \item We can use algorithm to detect cycles; just check if gray
		%       node is ever met.
	\end{itemize}
	\item If graph is not connected we have to try to start the traversal
	at all nodes. \\
	
	\fbox{
		\parbox{8cm}{ 
			FOR each vertex $u \in V $ DO
			\begin{itemize}
				\item[] IF color[$u$] = white THEN BFS($u$)
			\end{itemize}
			OD
	}}
	
	\begin{itemize}
		\item Note: We can use algorithm to compute connected components in
		$O(|V|+|E|)$ time.
	\end{itemize}
\end{itemize}



\subsection{Depth-first search (DFS)}
\begin{itemize}
	\item If we use stack instead of queue $Q$ we get another traversal order;
	depth-first
	\begin{itemize}
		\item We go ``as deep as possible'',
		\item Go back until we find unexplored adjacent vertex,
		\item Go as deep as possible,
		
		$\vdots$
	\end{itemize}
	\item Often we are interested in ``start time'' and ``finish time'' of
	vertex $u$
	\begin{itemize}
		\item {\em Start time\/} (d[$u$]): indicates at what ``time'' vertex
		is first visited.
		\item {\em Finish time\/} (f[$u$]): indicates at what ``time'' all
		adjacent vertices have been visited.
	\end{itemize}
	\item We can write DFS iteratively using the same algorithm as for BFS
	but with a STACK instead of a QUEUE, or, we can write a recursive DFS
	procedure
	\begin{itemize}
		\item We will color a vertex gray when we first meet it and black
		when we finish processing all adjacent vertices.
	\end{itemize}
	\item Algorithm: \\
	
	\fbox{
		\parbox{8cm}{
			{\sc DFS($u$) } 
			\begin{itemize}
				\item[] color[$u$] = gray
				\item[] $d[u]$ = time
				\item[] time = time + 1
				\item[] FOR $(u,v)\in E$ DO
				\begin{itemize}
					\item[] IF color[$v$] = white THEN
					\begin{itemize}
						\item[] parent[$v$] = $u$
						\item[] {\sc DFS(v)}
					\end{itemize}
					\item[] FI
				\end{itemize}
				\item[] OD
				\item[] color[$u$] = black
				\item[] $f[u]$ = time
				\item[] time = time + 1
			\end{itemize}
	}}
	
	\item Algorithm runs in $O(\vert V \vert + \vert E \vert)$ time
	\begin{itemize}
		\item As before we can extend algorithm to unconnected graphs and
		we can use it to detect cycles in $O(|V|+|E|)$ time.
	\end{itemize}
	
	\item As previously parent[$v$] forms a tree; {\em DFS-tree}
	\begin{itemize}
		\item Note: If $u$ is descendent of $v$ in DFS-tree  then $d[v] <
		d[u] < f[u] < f[v]$
	\end{itemize}
\end{itemize}

%%%%%%%%%%%%%%%%%%%%%%%%%%%%%%%%%%%%%%%%%%%%%%%%%%%%%%%%%%%%%%%%%%%%%%


\section{Topological sorting}
\begin{itemize}
	\item Definition: Topological sorting of {\em directed acyclic graph}
	$G=(V,E)$ is a linear ordering of vertices $V$ such that $(u,v) \in E
	\Rightarrow u$ appear before $v$ in ordering.
	\item Topological ordering can be used in scheduling:
	\begin{itemize}
		\item Example: Dressing (arrow implies ``must come before'') \\ \\
		% \centerline{\epsfig{file=top_sort_example.eps,height=5.5cm}} \\
		
		We want to compute order in which to get dressed. One possibility: \\
		
		
		\vspace*{\baselineskip}
		
		The given order is one possible topological order.
	\end{itemize}
	\item Algorithm: Topological order just reverse DFS finish time ($
	\Rightarrow O(\vert V \vert + \vert E \vert)$ running time).
	\item Correctness: $(u,v) \in E \Leftrightarrow f(v) < f(u)$
	\begin{itemize}
		\item Proof: When $(u,v)$ is explored by DFS algorithm, $v$ must be
		white or black (gray $\Rightarrow$ cycle).
		\begin{itemize}
			\item $v$ white: $v$ visited and finished before $u$ is
			finished $\Rightarrow f(v) < f(u)$
			\item $v$ black: $v$ already finished $\Rightarrow f(v) <
			f(u)$
		\end{itemize}
	\end{itemize}
	\item Alternative algorithm: Count in-degree of each vertex and repeatedly
	number and remove in-degree 0 vertex and its outgoing edges: Homework. 
	
\end{itemize}

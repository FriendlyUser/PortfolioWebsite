\chapter{ELEC 360 CheatSheet}
Some keywords used can be found in the glossary section including:
\begin{itemize}
	\item  \gls{partFrac}, \gls{openLoop}, \gls{SISO}, \gls{MIMO}, and \gls{LTI}
	\item  \gls{conSys}, \gls{closedLoop}, \gls{DC Motors}, and \gls{Op Amps}
\end{itemize}

\begin{multicols}{3}

%\includegraphics[width=1.77165in,height=0.95669in]{media/image1.png}

\section{LAPLACE TRANSFORMS}

Final Value Theorem

In Control Engineering, the Final Value Theorem is used most frequently to determine the steady-state value of a system. The real part of the poles of the function must be < 0.
\begin{align}
& \lim\limits_{t \rightarrow \infty} x(t) = \lim\limits_{t \rightarrow 0} sX(s)
\end{align}
%\includegraphics[width=1.77165in,height=0.92126in]{media/image2.png}

%\includegraphics[width=1.77165in,height=1.14961in]{media/image3.png}

Initial Value Theorem

\begin{align}
& \lim\limits_{t \rightarrow 0} x(t) = \lim\limits_{t \rightarrow \infty} sX(s)
\end{align}
%\includegraphics[width=1.77165in,height=0.90551in]{media/image4.png}

%\includegraphics[width=1.77165in,height=0.77953in]{media/image5.png}

\section{SOLUTION OF LINEAR DIFFERENTIAL EQUATION}

\[A\ddot{y} + B\dot{y} + Cy = u\left( t \right)\]

\[\text{with initial conditions }\dot{y}\left( 0 \right)\ and \ y(0)\]

\[\text{is solved by constructing the equation}\]

%\[A\left\lbrack s^{2}Y\left( s \right) - sy\left( 0 \right) - \dot{y}\left( 0 \right) \right\rbrack + B\left\lbrack \text{sY}\left( s \right) - y\left( 0 \right) \right\rbrack + CY(s)\]

\begin{align*}
 &A\left\lbrack s^{2}Y\left( s \right) - sy\left( 0 \right) - \dot{y}\left( 0 \right) \right\rbrack + B\left\lbrack \text{sY}\left( s \right) - y\left( 0 \right) \right\rbrack \\
 & + CY(s)
\end{align*}
\[that\ is\ to\ say:\]

\[\mathcal{L}\left\{ \ddot{y} \right\} = s^{2}Y\left( s \right) - sy\left( 0 \right) - \dot{y}(0)\]

\[\text{and}\]

\[\mathcal{L}\left\{ \dot{y} \right\} = sY\left( s \right) - y\left( 0 \right)\]

\section{STATESPACE REPRESENTATIONS}

To generate from a differential equation:

Use the \textbf{closed loop transfer function}

\[\frac{Y\left( s \right)}{U\left( s \right)} = \frac{s + A}{s^{3} + \text{Bs}^{2} + s + A}\]

Separate and take the inverse Laplace transform

\begin{align*} &s^{3}Y\left( s \right) + \text{Bs}^{2}Y\left( s \right) + sY\left( s \right) + AY\left( s \right) \\
 &= sU\left( s \right) + AU\left( s \right)
 \end{align*}

\[= \dddot{y} + B\ddot{y} + \dot{y} + Ay = \dot{u} + Au\]

Then define state variables:

\(x_{1} = y\) \(\dot{x_{1}} = \dot{y} = x_{2}\)

\(x_{2} = \dot{y}\) \(\dot{x_{2}} = \ddot{y} = x_{3}\)

\(x_{3} = \ddot{y}\)
\(\dot{x_{3}} = \dddot{y} = \dot{u} + Au - B\ddot{y} - \dot{y} - Ay\)

Then construct the state space matrix

\[\begin{bmatrix}
\dot{x_{1}} \\
\dot{x_{2}} \\
\dot{x_{3}} \\
\end{bmatrix} = \begin{bmatrix}
0 & 1 & 0 \\
0 & 0 & 1 \\
 - A & - 1 & - B \\
\end{bmatrix}\begin{bmatrix}
x_{1} \\
x_{2} \\
x_{3} \\
\end{bmatrix} + \begin{bmatrix}
\beta_{1} \\
\beta_{2} \\
\beta_{3} \\
\end{bmatrix}u\]

and

\[y = \begin{bmatrix}
1 & 0 & 0 \\
\end{bmatrix}\begin{bmatrix}
x_{1} \\
x_{2} \\
x_{3} \\
\end{bmatrix} + \beta_{0}u\]

\(\beta\) values can be calculated as follows:

\[\beta_{0} = b_{0}\]

\[\beta_{1} = b_{1} - a_{1}\beta_{0}\]

\[\beta_{2} = b_{2} - a_{1}\beta_{1} - a_{2}\beta_{0}\]

\[\beta_{3} = b_{3} - a_{1}\beta_{2} - a_{2}\beta_{1} - a_{3}\beta_{0}\]

Where the values for \(a_{x}\text{\ and\ }b_{x}\)come from:

\[\dddot{y} + a_{1}\ddot{y} + a_{2}\dot{y} + a_{3}y = b_{0}\dddot{u} + b_{1}\ddot{u} + b_{2}\dot{u} + b_{3}u\]


\[\ddot{x}=(F-c\dot{x_{1}}-kx)/m\]
where
	x=position,$\dot{x}$=speed/velocity,$\ddot{x}$=acceleration
	c = damping constant,
	m = mass,
	F = force
	k = spring constant,

To perform inverse (find transfer function from statespace model)

\[G\left( s \right) = d + c{(sI - A)}^{- 1}b\]


where

\[\begin{bmatrix}
\dot{x_{1}} \\
\dot{x_{2}} \\
\dot{x_{3}} \\
\end{bmatrix} = \begin{bmatrix}
A_{11} & A_{12} & A_{13} \\
A_{21} & A_{22} & A_{23} \\
A_{31} & A_{32} & A_{33} \\
\end{bmatrix}\begin{bmatrix}
x_{1} \\
x_{2} \\
x_{3} \\
\end{bmatrix} + \begin{bmatrix}
b_{1} \\
b_{2} \\
b_{3} \\
\end{bmatrix}u\]

And

\[y = \begin{bmatrix}
c_{1} & c_{2} & c_{3} \\
\end{bmatrix}\begin{bmatrix}
x_{1} \\
x_{2} \\
x_{3} \\
\end{bmatrix} + \lbrack d\rbrack u\]

The inverse of a square 2x2 matrix is found by:

\[\begin{bmatrix}
a & b \\
c & d \\
\end{bmatrix}^{- 1} = \frac{1}{ad - bc}\begin{bmatrix}
d & - b \\
 - c & a \\
\end{bmatrix}\]

The inverse of a square 3x3 matrix is found by:
%\includegraphics[width=2.52147in,height=1.02367in]{media/image8.png}

For a 3×3 matrix
$$A=\begin{bmatrix}
	a_{11} & a_{12}  & a_{13} \\
	a_{21} & a_{22}  & a_{23}\\
	a_{31} & a_{32}  &  a_{33}
\end{bmatrix}$$	
the matrix inverse is:

$$A^{-1}= \frac{1}{|A|}\begin{bmatrix}
\begin{bmatrix} a_{22} & a_{23} \\ a_{32} & a_{33}\end{bmatrix} & \begin{bmatrix} a_{13} & a_{12} \\ a_{33} & a_{32}\end{bmatrix} & \begin{bmatrix} a_{12} & a_{13} \\ a_{22} & a_{23}\end{bmatrix} \\
\begin{bmatrix} a_{23} & a_{21} \\ a_{33} & a_{31}\end{bmatrix}& \begin{bmatrix} a_{11} & a_{13} \\ a_{31} & a_{33}\end{bmatrix} & \begin{bmatrix} a_{13} & a_{11} \\ a_{23} & a_{21}\end{bmatrix} \\
\begin{bmatrix} a_{21} & a_{22} \\ a_{31} & a_{32}\end{bmatrix} & \begin{bmatrix} a_{12} & a_{11} \\ a_{32} & a_{31}\end{bmatrix} & \begin{bmatrix} a_{11} & a_{12} \\ a_{21} & a_{22}\end{bmatrix}	
\end{bmatrix}$$
Solving a three order polynomial, without a fancy calculator
\begin{align*}
& x = \left[q + \left[ q^2 + (r-p^2)^3\right ]^{1/2} \right]^{1 /3} \\
& \left[q - \left[ q^2 + (r-p^2)^3\right ]^{1/2} \right]^{1 /3} + p
\end{align*}
Where 
\begin{align*}
& p = \frac{-b}{3a}, \ \ q =p^3 + \frac{bc-3ad}{3a^2}, \ \ r =\frac{c}{3a}
\end{align*}
\section{SECOND ORDER SYSTEMS}

\[G\left( s \right) = \frac{C(s)}{R(s)} = \frac{\omega_{n}^{2}}{s^{2} + 2\zeta\omega_{n}s + \omega_{n}^{2}}\]

\[K = \omega_{n}^{2};\ \ \ \ T = 2\zeta\omega_{n} = 2\sigma;\ \ \ \ \ \zeta = \frac{T}{2\sqrt{K}};\ \ \ \ \omega_{d} = \omega_{n}\sqrt{1 - \zeta^{2}}\]

\[\zeta = damping\ ratio; \sigma = real\ part\ of\ root;\]

\(\ \ \ \omega_{d} = damped\ natural\ frequency\)

\( omega_{n} = undamped\ natural\ frequency \)
%\includegraphics[width=1.96736in,height=1.37361in]{media/image9.png}

\[ Undamped:\ \zeta = 0;\] 
\[ Critically\ Damped: \zeta = 1 \] 
\[ Over\ Damped:\ \zeta > 1\]

Imaginary axis:

Frequency of oscillations

Real axis:

Decay time

\textbf{UNIT STEP RESPONSE OF A 2\textsuperscript{ND} ORDER UNDAMPED
SYSTEM}

$ t_{d}$ = delay time - to reach 50\% of $c\left( \infty \right) $\text{for the first time}.
$ t_{r}$ = rise time - time to reach 100 \% of $c\left( \infty \right)$ \text{for first time} . 
$ t_{p}$ = peak time - time to reach first peak. 
$ t_{s}$ = settling time - time to reach \& stay within 2\% or 5\% 
$ M_{p}$ = maximum overshoot. 
%\[{t_{d} = delay\ time - to\ reach\ 50\%\ of\ c\left( \infty \right)\text{for\ the\ first\ time}\backslash n}{t_{r} = rise\ time - time\ to\ reach\ 100\%\ of\ c\left( \infty \right)\text{for\ first\ time}\backslash n}{t_{p} = peak\ time - time\ to\ reach\ first\ peak\backslash n}{t_{s} = settling\ time - time\ to\ reach\ \&\ stay\ within\ 2\%\ or\ 5\%\backslash n}{M_{p} = maximum\ overshoot\ \left( \% \right)\backslash n}\]
\[{t_{r} = \frac{1}{\omega_{d}}\operatorname{}\left( - \frac{\omega_{d}}{\sigma} \right);\ \ \ \ t_{p} = \frac{\pi}{\omega_{d}}\backslash n}\]

\[{M_{p} = e^{- \frac{\zeta\omega_{n}\pi}{\omega_{d}}} = e^{- \frac{\eta \pi}{\sqrt{1 - \zeta^{2}}}} = e^{- \frac{\sigma \pi}{\omega_{d}}}}\]

\[t_{s} = \frac{4}{\sigma} = \frac{4}{\zeta\omega_{n}}\ \left( 2\%\ band \right)\]
\[t_{s} = \frac{3}{\sigma} = \frac{3}{\zeta\omega_{n}}\ \left( 5\%\ band \right)\]

Dominant poles are the ones closest to the imaginary axis

\section{ROUTH-HURWITZ STABILITY TEST}
\[a_0s^n + a_1s^{n-1}+ \cdots + a{n-1} s + a_n = 0\]
\[ \begin{array}{lllll}
\mbox{row n}   & a_0 & a_2 & a_4 & \cdots \\
\mbox{row n-1} & a_1 & a_3 & a_5 & \cdots \\
\mbox{row n-2} & b_1 & b_2 & b_3 & \cdots \\
\mbox{row n-3} & c_1 & c_2 & c_3 & \cdots \\
\cdots & \cdots & \cdots & \cdots & \cdots \\
\mbox{row 2} & * & * &  & \cdots \\
\mbox{row 1} & * &   &  & \cdots \\
\mbox{row 0} & * &   &  & \cdots \end{array} \]

\[ b_1=-\frac{det\left[\begin{array}{cc}a_0&a_2\\a_1&a_3\end{array}\right]}{a_1}
=\frac{a_1a_2-a_0a_3}{a_1} \]
\[ b_2=-\frac{det\left[\begin{array}{cc}a_0&a_4\\a_1&a_5\end{array}\right]}{a_1}
=\frac{a_1a_4-a_0a_5}{a_1} \]
\[ b_3=-\frac{det\left[\begin{array}{cc}a_0&a_6\\a_1&a_7\end{array}\right]}{a_1}
=\frac{a_1a_6-a_0a_7}{a_1} \]

\[ c_1=-\frac{det\left[\begin{array}{cc}a_1&a_3\\b_1&b_2\end{array}\right]}{b_1}
=\frac{b_1a_3-a_1b_2}{b_1} \]
\[ c_2=-\frac{det\left[\begin{array}{cc}a_1&a_5\\b_1&b_3\end{array}\right]}{b_1}
=\frac{b_1a_5-a_1b_3}{b_1} \]
\[ c_3=-\frac{det\left[\begin{array}{cc}a_1&a_7\\b_1&b_4\end{array}\right]}{b_1}
=\frac{b_1a_7-a_1b_4}{b_1} \]
\section{STEADY STATE ERROR ANALYSIS}

\[K_{p} = \operatorname{}{G\left( s \right)H(s)}\]

\[K_{v} = \operatorname{}{\text{sG}\left( s \right);\ \ \ \ K_{v} = \operatorname{}{s\left( \text{KG}\left( s \right) \right)}}\]

\[K_{a} = \operatorname{}{s^{2}G\left( s \right);\ \ \ \ K_{a} = \operatorname{}{s^{2}\left( \text{KG}\left( s \right) \right)}}\]

The type of system is determined by the number of poles at the origin.
For example:

%\includegraphics[width=2.60736in,height=0.52328in]{media/image10.png}

\section{ROOT LOCUS}

Root Locus presents the poles of the closed loop system when the gain K
changes from zero to infinity.

\textbf{Construction of the Root Locus}

Open loop transfer function

\[\text{KH}\left( s \right)G\left( s \right) = K\frac{B(s)}{A(s)}\]

m: the order of the \textbf{open-loop} numerator polynomial

n: the order of the \textbf{open-loop} denominator polynomial

\textbf{Rule 1:} number of branches equals the number of poles of the
open-loop transfer function

\textbf{Rule 2:} If the total number of poles and zeros of the open-loop
system to the right of the s-point on the real axis is odd, then this
point lies on the locus.

\textbf{Rule 3:} The locus starting point (K=0) are at the open-loop
poles and the locus ending points (K=$\infty$) are at the open loop zeros and
n-m branches terminate at infinity.

\textbf{Rule 4:} Slope of asymptotes of root locus as `s' approaches
infinity

\textbf{Rule 5:} Abscissa of the intersection between asymptotes of root
locus and real-axis.

%\includegraphics[width=2.13497in,height=0.89069in]{media/image11.png}

\textbf{Rule 6:} Break-away and break-in points. From the characteristic
equation

\[f\left( s \right) = A\left( s \right) + KB\left( s \right) = 0\ \ \ \ and\ \ \ \ K = - \frac{A\left( s \right)}{B\left( s \right)}\]

The break-away and break-in points can be found from

\[\frac{\text{dK}}{\text{ds}} = - \frac{A^{'}\left( s \right)B\left( s \right) - A\left( s \right)B^{'}\left( s \right)}{B^{2}\left( s \right)} = 0\]

\textbf{Rule 7:} Angle of departure from complex poles or zeros.
Subtract from 180° the sum of all angles from all other zeros and poles
of the open-loop system to the complex pole (or zero) with appropriate
signs.

\textbf{Rule 8:} Imaginary-axis crossing points. Use Ruth-Hurwitz table
to find value of K where system becomes unstable.

\section{BODE DIAGRAMS}

%\includegraphics[width=2.08800in,height=1.27279in]{media/image12.png}

\subsection{1. Gain Factor K:} Horizontal straight line at magnitude:
\(20\log{(K)}\text{dB}\)

Phase is zero.

\subsection{2. Integral or derivative factors}
\(\mathbf{(j\omega)}^{\mathbf{\pm 1}}\)

\[\left( \text{j}\omega \right)^{- 1}\  \rightarrow 20\log{\left| \frac{1}{\text{j}\omega} \right| = - 20\log\omega}\]

Magnitude: strait line with slope -20 dB/decade

Phase: -90°

\[\left( \text{j}\omega \right) \rightarrow 20\log\left| \text{j}\omega \right| = 20\log\omega\]

Magnitude: straight line with slope 20dB/decade

Phase: +90°

\subsection{3. First Order Factors}
\(\left( \mathbf{1 + j\omega T} \right)^{\mathbf{\pm 1}}\)

\begin{align}
& \left( 1 + j\omega T \right)^{- 1} \rightarrow 20\log\left| \frac{1}{1 + j\omega T} \right| \\ \notag
&= - 20\log\sqrt{1 + \omega^{2}T^{2}}\ \lbrack dB\rbrack
\end{align}

Approximation for Magnitude:

\[For\ \omega\ between\ 0\ and\ \frac{1}{T}\  \rightarrow 0dB\]
\[For\ \omega\  \gg \frac{1}{T} \  \rightarrow - 20dB/decade\]

Phase:

\[\omega = 0\  \rightarrow \varphi = 0\]
\[\omega = \frac{1}{T} \rightarrow \varphi = - 45\]
\[\omega = \infty \rightarrow \varphi = - 90 \]

\[\left( \mathbf{1 + j\omega T} \right)^{\mathbf{+ 1}}\]

\subsection{4. Quadratic Factors}

\[G\left( \text{j}\omega \right) = \frac{1}{1 + 2\zeta\left( \frac{\omega}{\omega_{n}} \right) + \left( \frac{\text{j}\omega}{\omega_{n}} \right)^{2}}\ \ ;\ \ 0 < \zeta < 1\]

Approximation for magnitude:

\[\omega \ll \omega_{n} \rightarrow 0dB\]

\[\omega \gg \omega_{n} \rightarrow - 20\log\left( \frac{\omega^{2}}{\omega_{n}^{2}} \right) = - 40\log{\left( \frac{\omega}{\omega_{n}} \right)\text{dB}}\]

\[Phase:\]

\[\omega = 0\  \rightarrow \varphi = 0\]
\[\frac{\omega}{\omega_{n}} = 1 \rightarrow \varphi = - 90\backslash \]
\[\omega = \infty \rightarrow \varphi = - 180 \]

\[Resonant\ Frequency:\]
	\[\omega_{r} = \omega_{n}\sqrt{1 - 2\zeta^{2}}\ \ for\ 0 < \zeta < 0.707\]

\[Resonant\ Peak\ Value:\]
\[M_{r} = \left| G\left( \text{j}\omega \right) \right|_{\max} = \frac{1}{2\zeta\sqrt{1 - \zeta^{2}}} \ for\ 0 < \zeta < 0.707 \]

Consider

\[ G_{1}\left( s \right) = \frac{1}{1 + Ts} \ G_{2}\left( s \right) = \frac{1}{1 - Ts} \ G_{3}\left( s \right) = \frac{1}{Ts - 1} \]

Then\ldots{}

\[\left| G_{1}(j\omega) \right| = \left| G_{2}(j\omega) \right| = \left| G_{3}(j\omega) \right|\]

And\ldots{}

\[\angle G_{2}\left( \text{j}\omega \right) = - \angle G_{1}\left( \text{j}\omega \right) and \]
\[ \angle G_{3}\left( \text{j}\omega \right) = 180 - \angle G_{1}\left( \text{j}\omega \right)\]

\[+ 90\ and\ \angle G_{3}\left( \text{j}\omega \right)\ goes\ from - 180\ to - 90\]


Generate based on the Bode Plot.

\textbf{The Nyquist Stability Criterion:} relates the stability of the
closed loop system to the frequency response of the open loop system.

\[Z = N + P\]

\textbf{Z:} Number of zeros of $(1+H(S)G(s))$ in the right half plane =
number of unstable poles of the closed-loop system.

\textbf{N:} Number of clockwise encirclements of the point $-1+j0$.

\textbf{P:} Number of poles of $G(s)H(s)$ in the right half plane.

If the plot makes a counter-clockwise encirclement of the $-1+j0$ point
then N becomes -1.

If Z = 0 the closed loop system is stable. If Z \textgreater{} 0 the
closed loop system has Z unstable poles. If Z \textless{} 0 a mistake
has been made and the calculations need to be rechecked.

\section{PHASE AND GAIN MARGINS}

A measure for relative stability of the closed-loop system is how close
\(G(j\omega)\), the frequency response of the open-loop system, comes to
the point $-1+j0$. This is represented by the phase and gain margins.

\textbf{Phase Margin:} The amount of additional phase lag at the Gain
Crossover Frequency \(\omega_{0}\) required to bring the system to the
verge of instability.

Gain crossover frequency:
\(\omega_{0}\text{\ for\ which\ }\left| G\left( j\omega_{0} \right) \right| = 1\)

Phase margin:
\(\gamma = 180 + \angle G\left( j\omega_{0} \right) = 180 + \phi\)

\textbf{Gain Margin:} The reciprocal of the magnitude
\(\left| G(j\omega_{1}) \right|\) at the Phase crossover frequency
\(\omega_{1}\) required to bring the system to the verge of instability.

Phase crossover frequency:
\(\omega_{1}\ where\ \angle G\left( j\omega_{1} \right) = - 180\)

Gain margin:

\[K_{g} = \frac{1}{\left| G(j\omega_{1}) \right|}\]

\[K_{g} = - 20\log\left| G\left( j\omega_{1} \right) \right|\]

\[K_{g}\ in\ dB > 0 = stable\] for minimum phase systems. 
	\[K_{g}in\ dB < 0\] = unstable for minimum phase systems. 

\textbf{Minimum phase systems:} all poles and zeros are in the left half
plane.

If the open-loop system is minimum phase and has both phase and gain
margins positive then the closed-loop system is stable.

For good relative stability both margins are required to be positive.

Good values for minimum phase system are:

Phase Margin: 30°-60°

Gain Margin: above 6dB


\end{multicols}